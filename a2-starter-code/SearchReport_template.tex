\documentclass{article}
\usepackage{graphicx} % Required for inserting images
\usepackage{makecell}

\title{CSE 415 Assignment 2 Report: \\
Evaluating Search Algorithms and Heuristics}
\author{Your Names Here }
\date{October 2025}

\begin{document}

\maketitle

\section{Introduction}
We are Deveshi Modi and Sushma Shankar.
This is our report for Assignment 2 covering both blind search algorithms
and heuristic search.

\section{Report on Part A: Problem Formulation and Blind Search Algorithms}

\subsection{Part A Step 4 (c)}

The 4th "return False" statement in the can_move method of HumansRobotsFerry.py is on line 87:

\begin{verbatim}
if h_remaining > 0 and h_remaining < r_remaining: return False
\end{verbatim}

This line implements the constraint that prevents humans from being outnumbered by robots on the departure side (the side where the ferry is currently located). Specifically, it checks if there are any humans remaining on the departure side (`h_remaining > 0`) and if the number of remaining humans is less than the number of remaining robots (`h_remaining < r_remaining`). If both conditions are true, it returns False, preventing the move from being executed. This ensures that humans are never left alone and outnumbered by robots on either side of the creek, which would violate the safety constraint of the puzzle.


\subsection{Part A Step 8}

The paths are not required in the report for the entries marked "skip."

{\flushleft
\begin{tabular}{|l|p{2cm}|p{2cm}|p{3cm}|}
\hline
\parbox{3.5cm}{Problem and\\ Algorithm} & Path Found & Path length & \#Nodes Expanded \\
\hline
\makecell[l]{Humans, Robots\\ and Ferry / DFS} & (skip) & & \\
\hline
\makecell[l]{Humans, Robots\\ and Ferry / BreadthFS} & Yes & 7 & 10 \\
\hline
\makecell[l]{Farmer, Fox, Chicken\\ and Grain/ DFS} & Yes & 7 & 7 \\
\hline
\makecell[l]{Farmer, Fox, Chicken\\ and Grain/ BreadthFS} & Yes & 7 & 9 \\
\hline
\makecell[l]{4-Disk Towers of\\ Hanoi/DFS} & (skip) & & \\
\hline
\makecell[l]{4-Disk Towers of\\ Hanoi/BreadthFS} & Yes & 15 & 70 \\
\hline
\end{tabular}}

\subsection{Part A Step 8, Path details}
 Paths found (if not shown in the table).  Copy the state sequences
 obtained from the search algorithm on the requested problems.

 \begin{itemize}
 \item HRF/BreadthFS: 
 H on left:3, R on left:3, ferry on left → 
 H on left:2, R on left:2, ferry on right → 
 H on left:3, R on left:2, ferry on left → 
 H on left:0, R on left:2, ferry on right → 
 H on left:2, R on left:2, ferry on left → 
 H on left:1, R on left:1, ferry on right → 
 H on left:3, R on left:1, ferry on left → 
 H on left:0, R on left:1, ferry on right → 
 H on left:1, R on left:1, ferry on left → 
 H on left:0, R on left:0, ferry on right
 \item FFCG/DFS: 
 Farmer:1, Fox:1, Chicken:1, Grain:1, Boat:LEFT → 
 Farmer:0, Fox:1, Chicken:0, Grain:1, Boat:RIGHT → 
 Farmer:1, Fox:1, Chicken:0, Grain:1, Boat:LEFT → 
 Farmer:0, Fox:0, Chicken:0, Grain:1, Boat:RIGHT → 
 Farmer:1, Fox:0, Chicken:1, Grain:1, Boat:LEFT → 
 Farmer:0, Fox:0, Chicken:1, Grain:0, Boat:RIGHT → 
 Farmer:1, Fox:0, Chicken:1, Grain:0, Boat:LEFT → 
 Farmer:0, Fox:0, Chicken:0, Grain:0, Boat:RIGHT
 \item FFCG/BreadthFS: Same path as DFS (optimal solution)
 \item 4-Disk TOH/BreadthFS: 
 [[4,3,2,1],[],[]] → [[4,3,2],[1],[]] → [[4,3],[1],[2]] → [[4,3],[],[2,1]] → [[4],[3],[2,1]] → [[4,1],[3],[2]] → [[4],[3,1],[2]] → [[4,2],[3,1],[]] → [[4,2,1],[3],[]] → [[4,2],[3],[1]] → [[4],[3,2],[1]] → [[4,1],[3,2],[]] → [[4],[3,2,1],[]] → [[],[3,2,1],[4]] → [[1],[3,2],[4]] → [[],[3,2],[4,1]] → [[2],[3],[4,1]] → [[2,1],[3],[4]] → [[2],[3,1],[4]] → [[],[3,1],[4,2]] → [[1],[3],[4,2]] → [[],[3],[4,2,1]] → [[3],[],[4,2,1]] → [[3,1],[],[4,2]] → [[3],[1],[4,2]] → [[3,2],[1],[4]] → [[3,2,1],[],[4]] → [[3,2],[],[4,1]] → [[3],[2],[4,1]] → [[3,1],[2],[4]] → [[3],[2,1],[4]] → [[],[2,1],[4,3]] → [[1],[2],[4,3]] → [[],[2],[4,3,1]] → [[2],[],[4,3,1]] → [[2,1],[],[4,3]] → [[2],[1],[4,3]] → [[],[1],[4,3,2]] → [[1],[],[4,3,2]] → [[],[],[4,3,2,1]]
 \end{itemize}

 \subsection{Part A Step 8,  Explanations of Certain Differences, Using Towers-of-Hanoi  }

\begin{paragraph}
(i. Why the maximum length of the OPEN list is more for one algorithm than the other)

For the 4-Disk Towers of Hanoi problem, BFS has a maximum OPEN list length of 16, while DFS has a maximum OPEN list length of 7. This difference occurs because BFS explores all nodes at the current depth level before moving to the next level. In the Towers of Hanoi problem, there are many possible states at each level, and BFS keeps all unexplored nodes at the current level in the OPEN list. DFS, on the other hand, explores one path deeply before backtracking, so it typically has fewer nodes in the OPEN list at any given time.

\end{paragraph}
\begin{paragraph}
(ii. Why the solution PATH length is different for one algorithm from that of the other.)

For the 4-Disk Towers of Hanoi problem, BFS found a solution with 15 edges, while DFS found a solution with 40 edges. This difference occurs because BFS guarantees finding the shortest path (minimum number of moves) to the goal, while DFS explores paths in depth-first order and may find a longer solution path. The Towers of Hanoi problem has an optimal solution of 15 moves $(2^n - 1)$ for n disks, which BFS correctly found. DFS, being uninformed about optimality, found a longer path by exploring deeper into the search space before finding the goal.

\end{paragraph}

% -----------------------------
\newpage
\section{Report on Part B: Heuristics for the Eight Puzzle}

(Your results for Part B should be reported in the table below.)


\subsection{Results with Heuristics for the Eight Puzzle}

{\flushleft
\begin{tabular}{|c|l|c|l|c|c|c|}
\hline
Puzzle & Heuristic & Solved? & \# Soln Edges & Soln Cost & \# Expanded & Max Open\\
\hline
A & none (UCS) & Y & 7 & 7.0 & 7 & 6 \\
\hline
A & Hamming & Y & 7 & 7.0 & 7 & 6 \\
\hline
A & Manhattan & Y & 7 & 7.0 & 7 & 6 \\
\hline
A & Simple Misplaced & Y & 7 & 7.0 & 7 & 6 \\
\hline
B & none (UCS) & Y & 12 & 12.0 & 33 & 25 \\
\hline
B & Hamming & Y & 12 & 12.0 & 33 & 25 \\
\hline
B & Manhattan & Y & 12 & 12.0 & 33 & 25 \\
\hline
B & Simple Misplaced & Y & 12 & 12.0 & 33 & 25 \\
\hline
C & none (UCS) & Y & 14 & 14.0 & 20 & 15 \\
\hline
C & Hamming & Y & 14 & 14.0 & 20 & 15 \\
\hline
C & Manhattan & Y & 14 & 14.0 & 20 & 15 \\
\hline
C & Simple Misplaced & Y & 14 & 14.0 & 20 & 15 \\
\hline
D & none (UCS) & Y & 16 & 16.0 & 7982 & 4700 \\
\hline
D & Hamming & Y & 16 & 16.0 & 589 & 368 \\
\hline
D & Manhattan & Y & 16 & 16.0 & 148 & 96 \\
\hline
D & Simple Misplaced & Y & 16 & 16.0 & 589 & 368 \\
\hline

\end{tabular} }

\begin{verbatim}
Puzzle A: [3,0,1,6,4,2,7,8,5]
Puzzle B: [3,1,2,6,8,7,5,4,0]
Puzzle C: [4,5,0,1,2,8,3,7,6]
Puzzle D: [0,8,2,1,7,4,3,6,5]
\end{verbatim}

\subsection{Evaluating Our Custom Heuristics}

\subsubsection{Simple Misplaced Tiles Heuristic Description}
Our custom heuristic is a simple extension of the Hamming distance heuristic. It counts misplaced tiles (excluding the blank) and adds a penalty of 1 if the blank tile is not in the center position (1,1). This heuristic is very easy to compute and provides slightly better guidance than basic Hamming distance.

\subsubsection{Underlying Intuition}
The intuition behind this simple heuristic is that having the blank tile in the center position (1,1) is strategically advantageous because it provides maximum flexibility for moving other tiles. The center position allows the blank to move in all four directions, making it easier to rearrange tiles. This small addition to the basic Hamming distance provides better guidance while remaining very simple to compute.

\subsubsection{Admissibility}
The Simple Misplaced Tiles heuristic is admissible because:
\begin{itemize}
\item Each misplaced tile requires at least 1 move to reach its correct position
\item The blank position penalty of 1 is a lower bound on the additional cost
\item The total heuristic value never exceeds the true optimal cost
\end{itemize}

\subsubsection{Computational Cost Comparison}
\begin{itemize}
\item \textbf{Hamming Distance}: O(n) - counts misplaced tiles
\item \textbf{Manhattan Distance}: O(n) - calculates distance for each tile
\item \textbf{Simple Misplaced}: O(n) - counts misplaced tiles plus one position check
\end{itemize}

The Simple Misplaced heuristic has the same computational complexity as Hamming distance but provides slightly better guidance by considering blank tile position.

\subsubsection{Performance Analysis}
The results show that:

\textbf{Puzzles A, B, C}: All heuristics performed identically, suggesting these puzzles have relatively simple solution paths where the additional guidance doesn't provide significant advantage.

\textbf{Puzzle D}: This puzzle demonstrates the power of informed heuristics:
\begin{itemize}
\item UCS: 7,982 states expanded (very inefficient)
\item Hamming: 589 states expanded (13.6x improvement)
\item Manhattan: 148 states expanded (53.9x improvement)
\item Simple Misplaced: 589 states expanded (13.6x improvement)
\end{itemize}

The Simple Misplaced heuristic performed identically to Hamming distance on Puzzle D, demonstrating that the blank position consideration provides minimal additional benefit for this particular puzzle configuration.

\subsection{Overall Performance Analysis}

The experimental results reveal several important insights about heuristic search performance:

\subsubsection{Heuristic Effectiveness Ranking}
Based on the number of states expanded across all puzzles:
1. \textbf{Manhattan Distance}: Best overall performance, especially on complex puzzles
2. \textbf{Hamming Distance}: Moderate improvement over UCS
3. \textbf{Simple Misplaced}: Similar performance to Hamming distance
4. \textbf{UCS (no heuristic)}: Least efficient, particularly on complex puzzles

\subsubsection{Key Observations}
\begin{itemize}
\item \textbf{Optimal Solutions}: All algorithms found optimal solutions (same path length and cost)
\item \textbf{Heuristic Guidance}: More informed heuristics significantly reduce search space
\item \textbf{Puzzle Complexity}: The benefit of better heuristics increases with puzzle difficulty
\item \textbf{Memory Usage}: Better heuristics also reduce maximum open list size
\end{itemize}

\subsubsection{Computational Trade-offs}
The Simple Misplaced heuristic demonstrates that even small improvements to basic heuristics can be valuable. While it doesn't provide dramatic performance gains over Hamming distance, it shows how simple modifications can enhance heuristic guidance without significant computational overhead.

\newpage
\section{Partnership Retrospective}

\subsection{Partnership?}
Did you work in a partnership? Yes.

If so, who were the partners? Deveshi Modi and Sushma Shankar.
\subsection{Collaboration}
Also if so, how did you did you divide the work of this assignment?\\
We worked together through all of the parts.
\subsection{Newness of the Collaboration}
If this was a new sort of experience for either of you, please mention that,
and in what way(s) it felt new. \\
We have worked together previously in CSE373.



\end{document}
